\documentclass[12pt, a4paper, roman]{moderncv}

\usepackage{paralist}
\usepackage[UTF8, scheme = plain, heading = false]{ctex}
\usepackage{geometry}

\moderncvstyle{classic}
\moderncvcolor{purple}

\geometry{scale = 0.75}
\AtBeginDocument{\settowidth{\hintscolumnwidth}{2020 年 -- 2023 年}}
\AtBeginDocument{\hypersetup{pdfstartview = FitH}}

\name{何}{景烨}
\title{个人简历}
\email{HJY@future-world.net}
\phone[mobile]{+86~130~400~8000}
\homepage{www.future-world.net}
\photo{1.jpg}

\begin{document}
\maketitle

\section{教育背景}
\cventry{2020 年 -- 2023 年}{深圳技术大学}{}{}{\textit{物联网工程}}{核心课程:一堆东西}

\section{计算机技能}
\cvitem{C/C++}{熟悉,学过}
\cvitem{Python}{熟悉,学过}
\cvitem{HTML}{不熟悉,在学}
\cvitem{\LaTeX}{不熟悉,在学}

\section{外语技能}
\cvitemwithcomment{英语}{熟练}{过六级}
\cvitemwithcomment{德语}{简单对话}{幼儿园水平}

\section{实践背景}
\subsection{校内实践}
\cventry{2022 年}{面向穿戴康复机器人应用的高精度低功耗sEMG肌电采集系统}{}{}{}{
    \begin{compactitem}
        \item 开发STM32F4嵌入式程序,使用ADC采集模拟前端芯片放大的人体表面肌肉电信号(sEMG);
        \item 通过DMA方式传输ADC数据,定时器中断确定采样率,实现对模拟信号的数字采集;
        \item 通过IIR滤波算法实现对工频信号的滤波处理;
        \item 使用SPI配置模拟前端芯片寄存器,实现改变芯片放大倍数功能;
        \item 通过UART将数据上传到上位机,实现肌电信号的模式识别功能。
    \end{compactitem}
}

\end{document}